\documentclass[a4paper]{article}

\usepackage[english]{babel}
\usepackage[utf8]{inputenc}
\usepackage{amsmath}
\usepackage{graphicx}
\usepackage[colorinlistoftodos]{todonotes}

\title{P2PBay}

\author{Henrique Rocha 68621\\
Ângelo Pingo 72413\\
Pedro Braz 73991}

\date{\today}

\begin{document}
\maketitle



\section{Introduction}
 P2PBay is a peer-to-peer application that allows clients to auction items over an internet-scale network, which is scalable, secure, supports highly volatile users and doesn't require a centralized server. 

\section{Requirements}


\section{Protocol Specification}

\subsection{Distributed Hash Table}
	Our protocol is going to work on top of a Kademlia DHT, which will take care of building the network and information routing. The specific library chosen is the java implementation TomP2P.

\subsubsection{TomP2P}
	TomP2P provides all Kademlia features, using the same 160 bit identifier space and a XOR based algorithm to define distance between nodes. It also provides us with Future objects that are used to make non-blocking requests, increasing peer availability when connected to the network.  
   \\\\--More technical stuff about tom
   
\subsubsection{TomP2P Shortcomings}
user auth \\ security \\ searching

   
\subsubsection{Other DHTs}
	--what is similar with other DHTs
	\\--what we could do with other DHTs
	\\--why every other choice sucks


\subsection{P2PBay}
	--how new user starts\\
    --how user login is done\\
    --how items are created\\
    --how to search\\
    --how to bid\\
    --how are differentiated (auth: and file:)\\ 
    --how to deal with concurrency\\
    --how to dead with documents timing out (republish)

\subsubsection{Protocol Testing}



\end{document}